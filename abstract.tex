\begin{abstract}
%Unfortunately, the expanded use of
%these devices also has lead to increased privacy and security 
%challenges, which makes research on these devices difficult.
%\albert{12 pages, plus unlimited refs. Abstract 250 words max. Due 2015-12-09.}
Due to their omnipresence, mobile devices such as smartphones 
and tablets could be of tremendous value to research. 
However, since research projects can use these devices to collect 
data that reveals personal information, there are substantial
privacy concerns with mobile device use. Therefore, researchers must go through 
a detailed IRB process before recruiting participants.
As such, many research studies either do not use data from real participants or use
data collected from a controlled subset of
participants. %often university students.
%such as investigating end-user traffic patterns, etc. 
%However, the expanded use of mobile 
%devices has resulted in increased privacy and security break-ins 
%on these devices. Therefore, many research institutions require that 
%research on mobile devices must be conducted in a responsible 
%and ethical manner. %challenging for researchers to 
%%conduct research on these devices without compromising
%%the privacy of device owners. 
%This poses challenges to the existing mobile testbeds that 
%play an important role in evaluating network research ideas. 
%The question remains whether research on 
%mobile devices can be carried out using a network testbed, without causing breaches of 
%personal privacy and device security. 

In this paper we present Sensibility Testbed, an experiment  
platform that lowers the policy and technical barriers to performing 
experiments on end users' mobile devices.
%for advancing the use of personal mobile devices in research. 
%security and privacy measures to ensure the 
%The use model of Sensibility Testbed is
%unique in that it (1) manages how device owners make their
%devices accessible to the research community, and (2) offers
%Through the use of a secure sandbox, 
Data can be gathered 
through Sensibility Testbed while reducing the risk of a privacy problem.
%Sensibility Testbed provides privacy protection of mobile device 
%data, and maintains the security of device systems from 
%potentially buggy experiment code. The experiment code in Sensibility 
%Testbed runs in isolated sandboxes to 
%prevent inadvertent or malicious bugs. 
Access to sensor data on a mobile device is mediated according to the policies 
set by the IRB of the researcher's 
institution. These policies are enforced by Sensibility Testbed via technical 
means, not just as stated policies, by a set of non-bypassable 
\textit{blurring layers}. 
Different policies can be created by customizing individual blurring layers 
and loading and stacking them in order on a device. 
These technical innovations simplify the IRB process and allow Sensibility 
Testbed to cater to a wider range of participants. Furthermore, 
Sensibility Testbed's platform design saves researchers' 
effort in recruiting participants and managing a deployment. 
%\cappos{The sentence before this needs to be tightened.  It mentions 
%all sorts of things we don't really cover in the paper.}
Our evaluation shows that \sysname %enables useful experiments to be run
%while 
limits the impact that a buggy or malicious experiment could have on user 
privacy.
%using a combination of blurring and rate-limiting. 
%beyond that of previous campus-only,
%incentivized testbeds, thereby adding to the diversity of the
%installed base in terms of participating devices, OS software
%versions, network operators, participant demographics, countries,
%etc.
%
%experiment measures to researchers that allow them to collect
%data from remote mobile devices without rendering these devices
%at risk. 
%					
%Sensibility Testbed provides privacy mechanisms to protect
%device owners' privacy by mediating data access according to
%researcher's IRB policies. The testbed also uses security
%measures to prevent any inadvertent or malicious bugs in
%experiment code and thus protects the devices. 

%Sensibility Testbed lowers the barriers for researchers to perform
%research experiment on end users' mobile devices, and 
%allows safe research without rendering devices vulnerable.

\end{abstract}
