\begin{abstract}
The research potential of personal devices, such as smartphones
and tablets, has sparked a number of recent research initiatives.
Because these devices are omnipresent, they could be of tremendous
value to the research community. Unfortunately, the expanded use of
these devices also has lead to increased privacy and security 
challenges, which makes research on these devices difficult.
In this paper we introduce Sensibility Testbed, an experimental  
platform for advancing the use of smart devices in research, 
%security and privacy measures to ensure the
while ensuring the security of the mobile devices and the privacy of
the data generated. 
%The use model of Sensibility Testbed is
%unique in that it (1) manages how device owners make their
%devices accessible to the research community, and (2) offers

Sensibility Testbed's platform design not only encourages reusing 
experiment code and infrastructure, but also saves effort on 
recruiting participants and managing a deployment. 
Furthermore, Sensibility Testbed maintains device safety and privacy of 
participants. Experiment code runs in isolated sandboxes to 
prevent any inadvertent or malicious bugs. Access 
to sensor values is mediated according to the policies 
set by the institutional review board of the researcher's 
university or institution. 
%using a combination of blurring and rate-limiting. 
This allows Sensibility Testbed to cater to a wider range of 
participants than prior mobile testbeds. 
%beyond that of previous campus-only,
%incentivized testbeds, thereby adding to the diversity of the
%installed base in terms of participating devices, OS software
%versions, network operators, participant demographics, countries,
%etc.
%
%experiment measures to researchers that allow them to collect
%data from remote mobile devices without rendering these devices
%at risk. 
%					
%Sensibility Testbed provides privacy mechanisms to protect
%device owners' privacy by mediating data access according to
%researcher's IRB policies. The testbed also uses security
%measures to prevent any inadvertent or malicious bugs in
%experiment code and thus protects the devices. 
Therefore, Sensibility
Testbed allows safe research without rendering devices
vulnerable, and lowers the barriers for researchers to perform
research experiment on end users' mobile devices.

\end{abstract}