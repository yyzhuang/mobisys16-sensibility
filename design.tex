\section{Sensibility Testbed Design}\label{sec-design}

This section describes the design of the Sensibility Testbed. 
%On one hand, Sensibility Testbed manages how device owners make their
%devices accessible to the research community. On the other hand,
%it offers technical resources to researchers that allow them to
%securely collect data from remote mobile devices. 
We start with
an overview of the testbed infrastructure (Section~\ref{sec-overview}), 
and description of each testbed component 
(Section~\ref{sec-component}), then finally discuss how all the 
components work together to allow researchers to conduct experiments,
and protect device owners' personal information
(Section~\ref{sec-scenario}).


\subsection{Overview}\label{sec-overview}

%\subsubsection{Interacting Parties}\label{sec-parties}
In Sensibility Testbed, there are three types of interacting
parties, as shown in Figure~\ref{fig-arch}: mobile \textit{devices} 
owned by ordinary people, with our app installed; a 
\textit{clearinghouse} server that discovers and configures
participating devices; and \textit{experimenters} who want to run
experiments on mobile devices. 

\begin{figure}
\center{\includegraphics[width=\columnwidth]{figs/arch.pdf}}
%\vspace*{-20pt}
\caption{\small Sensibility Testbed architecture. \label{fig-arch}}
\end{figure}

This section addresses how these elements interact to
enable safe experimentation on mobile
devices. It starts with a researcher providing his institute's IRB policies for accessing devices to the clearinghouse server

(Section~\ref{sec-ch}). \yanyan{need a screenshot} 
%These policies restrict what and how data can be accessed by the 
%researcher. 
The clearinghouse server helps the researcher acquire and manage
devices, and codifies the policies specified by the
researcher's IRB. 
%into data blurring layers that are enforced on
%mobile devices. Such a process can protect device
%owners' personal information. 
After obtaining remote devices, researchers can perform
experiments directly on the devices, using the credentials assigned by
the clearinghouse. Researchers' code runs in a sandbox 
that isolates the code from the rest of the device
host system (Section~\ref{sec-repy}). To control the execution of 
code, the researcher uses his own desktop or laptop computer to manage the 
experiments via an experiment manager. This tool can deploy 
and run experiments in sandboxes on remote devices that are 
acquired through the clearinghouse (Section~\ref{sec-emt}).

%\subsubsection{Enforcing IRB Approved Policies}


\subsection{Testbed Components}\label{sec-component}

The components of the Sensibility Testbed are shown in Figure~\ref{fig-arch}.
Each is critical to the operation of the testbed. 
In this section, we describe the function of each component: the
software running on a mobile device, the clearinghouse~\cite{ch}, 
and experiment manager.

\subsubsection{Device Software}\label{sec-repy}

%In Sensibility Testbed, all experiments execute in a secure 
%sandbox on the end devices.
%Experimenters' code executes in a sandbox that isolates the 
%experiment code from the device host system. 
Sensibility Testbed uses a sandbox called Repy (Restricted 
Python)\footnote{\scriptsize This is the 
same security-reviewed sandbox~\cite{cappos2010retaining} used in
our prior work, the Seattle testbed~\cite{seattle}. This sandbox
mitigates the impact of bugs in experimenter code.}, which 
provides security isolation and performance isolation on mobile devices.
%Instead of developing full-fledged Android apps, all 
%experiments in Sensibility Testbed are written in a language
%similar to Python, and run in a secure %Python-based
Experimenters use a Python-like programming interface~\cite{repyv2}
to write experiment code, upload the code and execute it in the
sandbox. The programming interface includes functions for networking, 
file system, threading, locking, logging, and so on. To access sensors, \lois("file system" does not sound like a function. Its the only term in this list of activities that is not a verb)
the sandbox also has a set of sensor functions~\cite{sensors}. 
Details about implementing the Repy sandbox for mobile 
devices are described in Section~\ref{sec-repy-ext}.

To install Repy and other software on a device, the owner downloads 
our Sensibility Testbed app from the Google Play Store \cite{sensibility-app}.
The app displays a consent form, \yanyan{cite link} containing the testbed's general usage policy. The device owner reviews
 \yanyan{cite link} and must agree to this policy before installation. If the device owner gives his
consent, the device will be installed with the Repy sandbox, the native Android code to 
start or stop the sandbox, and an interface to communicate with the testbed 
infrastructure, particularly the clearinghouse (described below). 
By agreeing to our general usage policy, device 
owners of different ages, from different countries, with different
background need only to opt into our testbed as 
volunteers \textit{once}, at the time of app installation). As a result, an 
experimenter who wants to conduct a research experiment 
%requests devices through our clearinghouse, which assigns 
%them devices from a set of available resources. As a result, 
%the researcher
does not need to get consent from each subject for each individual
experiment. The testbed thus greatly simplifies the process for both the 
device owners and experimenters. 

%However, the current Repy sandbox does not include calls to access sensors. 
%%To obtain the sensor data, we need to extend the sandbox. 
%The extended Repy sandbox that allows sensor access  
%will be described in Section~\ref{sec-repy-ext}.
Furthermore, the Repy sandbox allows us to change the 
behavior of its programming interface, and control the 
data gathered from the device to adapt to any IRB proscribed limits.
%For  an experiment
%that involves GPS location, a privacy policy might restrict the
%level of data granularity available to the experiment. For example, it can
%obfuscate GPS location such that it only identifies the center
%of the city that the device is located in, rather than the exact
%location. Using the same technique, 
To illustrate, the IP address of a device may be anonymized, 
the frequency to access GPS location can be constrained, and 
access to cameras can be disabled.
%Such privacy protection is a contribution of Sensibility Testbed, 
%which does not exist in any prior work. 
The details of policy implementation are presented in 
Section~\ref{sec-layer} and \ref{sec-nanny}.

\subsubsection{Clearinghouse}\label{sec-ch}
The clearinghouse~\cite{ch} is a testbed server that keeps 
track of devices and mediates experimenter access to the 
available devices. It allows experimenters to register 
accounts and share access to a common pool of devices.
The clearinghouse does so by looking up available devices, and assigning
them to a researcher's experiment account upon request. 
Its key role is to facilitate device sharing, 
which relieves individual experimenters from repeatedly 
recruiting devices for each experiment.

%The clearinghouse
%plays an intermediate role between the experimenter and 
%the device owner.
%As described in Section~\ref{sec-overview}, when an 
%experimenter registers at the clearinghouse, he
%needs to provide his IRB policies. These policies ensure that
%the researcher cannot conduct experiments to access data that
%extend beyond the experiment policy. The clearinghouse 
%translates and codifies each policy, and instructs the 
%sandboxes on remote devices to implement these policies. 
%When experiment code is running in the sandbox, the 
%policies will be applied to restrict %the precision of sensor 
%%data or the frequency to access 
%sensor access. 
In order to obtain an IRB approval, a researcher first fills out an experiment
registration form \yanyan{cite url or show screenshot} at the 
clearinghouse. The clearinghouse website shows 
a list of available sensors, and each of their required accuracy 
and access frequency.\lois(does it show the "sensors" apart from the devices?) After filling out this form, the researcher 
downloads the detailed information about Sensibility Testbed and 
several relevant forms, such as 
those addressing consent, terms of participation (for device owners) 
and terms of usage (for the researcher), and so on. The researcher then uses 
these forms as a template to complete the IRB registration with his institution.
These forms serve as a set of reference documents 
\yanyan{cite our docs} to make it easier for the researcher to file the necessary IRB paperwork with her institution.

After the application is submitted, the researcher's IRB may disagree with 
the initial experiment requirements. %For example, Bob wants to access cell 
%IDs in cellular networks, but his IRB disallows such data access. Bob then
In this case, the researcher will revise the experiment requirements, refile the paperwork, obtain IRB approval, and
submit the revised experiment registration form to the clearinghouse. Finally, the clearinghouse
parses the registration form, extracts each data accuracy and access 
frequency approved by the researcher's IRB, and assigns an experiment 
account to the researcher. The clearinghouse
has a default set of blurring layers for accuracy and access frequency for each sensor. However, once the researcher requests 
devices, the clearinghouse supplies the extracted data from the 
researcher's registration form as input parameters to
the blurring layers that will be instantiated on the requested devices. Details about
how each policy is implemented will be introduced in Section~\ref{sec-policy}.

This Sensibility Testbed
clearinghouse protocol for research plays a central role in
easing the device recruitment and experiment setup for experimenters, 
and ensures the enforcement
of privacy policies\footnote{\scriptsize The Sensibility Testbed Clearinghouse
protocol for research with human subjects has been approved by
the IRB at New York University (IRB \# 15-10751).}. 

\subsubsection{Experiment Manager}\label{sec-emt}

To run code remotely on mobile devices, a researcher will use an
experiment manager dowloaded to his own computer 
%which contains Bob's private key, \path{key.bob-priv}, 
to access the sandboxes on the remote devices assigned by the clearinghouse. 
This experiment manager provides a light-weight command line 
console~\cite{demo-kit} that allows direct access from the 
experimenter's local machine to a set of remote devices. 
The clearinghouse has no involvement after the experimenter deploys 
her study and runs the code, and it does not store any
data on the experimenter's behalf. After collecting the data he needs, the
experimenter can use the experiment manager to download data from the remote devices. 
Alternatively, the experimenter can set up his own server to store all 
the data\footnote{\scriptsize
If an experimenter stores data on his own server, he must use protective
measures to ensure that the data sent from the mobile devices is
properly encrypted, and the server storage cannot be tampered
with by any other parties. This is enforced by requiring the experimenter to register
his server by providing its certificate and URL to our
clearinghouse. Following receipt of this data, the clearinghouse will instruct the devices
accessible to the experimenter that all the sensor data collected should be
sent to this server. The sandboxes on these devices will issue
\texttt{HTTPS POST} using the server's certificate, and send encrypted
data to the experimenter's server. Guidelines for secure storages of this data are mandated by the experimenter's IRB. Otherwise, researchers are required to use a data 
store service we provide (a service called Sensevis~\cite{sensevis}, 
not shown in Figure~\ref{fig-arch}).

\smallskip
In summary, 
%Prior to running an experiment on Sensibility Testbed, a
%experimenter first fills out a form in plain text to describe the
%purpose of the research experiment. This experiment description
%is created at the Sensibility Testbed clearinghouse
%where the researcher indicates the type of data to be collected,
%how that data will be used and stored, and so on. 
%
%Once this information is collected from the researcher, the
%clearinghouse automatically generates a set of blurring layers
%that implements the experiment policy (Section~\ref{sec-policy}). In
%Sensibility Testbed, researchers can collect data from the
%sensors on the device, such as GPS, Bluetooth, battery
%information, accelerometer, light, and orientation,
%etc. The blurring layers we provide consist of
%data access restrictions, created in accordance with
%researcher's experiment description, by using the Sensibility
%Testbed's sandboxing technique 
%(Section~\ref{sec-repy})~\cite{cappos2010retaining}. These restrictions ensure that
%the researcher cannot conduct experiments to access data that
%extend beyond the experiment policy. 
%
%This Sensibility Testbed
%clearinghouse protocol for research plays a central role in
%easing the approval process of IRB, and ensures the enforcement
%of privacy policy\footnote{The Sensibility Testbed Clearinghouse
%protocol for research with human subjects has been approved by
%the IRB at New York University. \yanyan{might need a link to
%your approval letter or ref number}}. 
using Sensibility Testbed, device owners' privacy is protected
from any inadvertent or malicious attempt, and researchers 
are able to go through a streamlined process of device 
recruitment and experiment setup.

%the device owners do not need to give consents 
%multiple times, to each project of each researcher. 
In the following section, we present several usage scenarios to
demonstrate how device owners opt in as volunteers, and how
researchers are able to to conduct experiments on devices without compromising device
owners' privacy.

\subsection{Testbed Usage Scenarios}\label{sec-scenario}
%\yanyan{should this section go after policies?} \lois(After reviewing these this morning, I am not sure you need these scenarios at all. They seem to repeats examples we have already seen in previous sections. We already saw the procedures for Bob to fill out the form or for Alice to download the app. There might be a few bits and pieces of new information here that we could work into something else?such as the twbut I don't think we need these scenarios as a separate section.)
To demonstrate how Sensibility Testbed's components interact with
each other to assist experimenter's experimentation, this section will go
through several scenarios. In these scenarios, a smartphone owner, Alice,
participates in the testbed; a researcher, Bob, runs code on
Sensibility Testbed using a number of devices, including Alice's smartphone. Bob is studying cellular service
quality in major cities. As such, he needs location information
of individual devices, their cellular service provider, network
type (3G, 4G, LTE, etc.), and signal strength.

Note that in Sensibility Testbed, there are two types of keys. A device
owner has an \textit{identification key} to identify the app installed on a 
device. An experimenter has a pair of public/private \textit{authentication 
keys} to authenticate the experimenter with the clearinghouse and 
the set of devices he is permitted to access. \lois(this is the first mention of keys and access. It seems to me this belongs either in the overview section, or in section 4. I don't think it belongs here)

\textbf{Smartphone owner volunteers as a testbed participant.}
%\label{sec-owner-participate}
When Alice, a smartphone owner, signs on as a participant in
Sensibility Testbed, she first downloads and installs our Sensibility Testbed
app~\cite{sensibility-app}. %which currently supports Android devices.
%This app contains a testbed key \path{key.sensibility} used for 
%device discovery (similar to a device owner's identification key). 
%the Repy sandbox (Section~\ref{sec-repy}), 
%and native Android code to handle user interaction, communicate 
%with the clearinghouse, and so on. 
%Before installation, the app displays a
%consent form to Alice. If Alice
%agrees to the terms and policy, the app will be installed on her
%device. 
Once the app is started, Alice's device can be
discovered by the clearinghouse. Her device periodically contacts 
a lookup service to advertise itself to Sensibility Testbed. 
A lookup service is a distributed key-value store, such as a 
distributed hash table (DHT)~\cite{dht}, that 
allows one to retrieve values associated with keys and to associate 
keys with values. 

To keep track of Alice's device, the
clearinghouse periodically queries the lookup service to
discover any new devices. Once Alice's device is discovered, the
clearinghouse obtains its identification key \path{key.alice} generated
during installation, and stores this key. 
%clearinghouse uses a database that stores her device's unique
%identification key, \path{key.alice}, generated during installation. 
This key is not associated with Alice's or her
device's identity, but only the app's installation on the device. If
Alice uninstalls the Sensibility app, \path{key.alice} is
deleted at the clearinghouse, which effectively unlinks
her device from any metadata stored on the clearinghouse.
Instead of uninstalling, Alice may also choose to opt out of
individual experiments.\lois(I don't remember seeing this opt-out option discussed and I think this is important. There may be some particular experiments a person would prefer not to be part of and how they are able to opt out is important)

\textbf{Researcher registers experiment and provides IRB policies.}
%\label{sec-irb-policy}
To run code on Sensibility Testbed, experimenter Bob provides a
set of detailed IRB policies from his institution. The process of how
Bob obtains such policies is described in Section~\ref{sec-ch}.
Bob first fills out an experiment registration form and specifies that 
%what data can be accessed by a research experiment, at which 
%granularity or frequency such data can be accessed, how data 
%should be securely stored, and so on. \yanyan{cite register 
%experiment website url.}
his experiment can (1) read location information
from devices at the granularity of a city, (2) read accurate
cellular signal strength and network type, as well as
%but not allow access to information about 
cell IDs, and (3) get location and
cellular network data updates every 10 minutes. 
%Bob submits an
%experiment description for these requirements, which the
%clearinghouse will codify into policies that are later enforced
%on remote mobile devices (Section~\ref{sec-ch}).
Bob then uses this form, along with other forms downloaded 
from the clearinghouse, to apply for IRB approval at his institution.

After the application is submitted, Bob's IRB disagrees that his 
experiment should access cell IDs in cellular networks, but approves 
the other policies. He submits the revised experiment registration form
and his IRB approval, and obtains an experiment account at the
clearinghouse. Once his account is activated, Bob obtains his 
authentication keys assigned by the 
clearinghouse. These keys are to authenticate Bob to the 
clearinghouse and the set of devices that he has access to.
%\path{bob.public} and \path{bob.private}.
Next, Bob can request a number of devices from the clearinghouse.

\textbf{Researcher acquires device(s) and runs an experiment.}
%\label{sec-acquire-run}
%The above clearinghouse protocol ensures the enforcement of data
%access policies. Additionally, 
To perform an experiment, Bob needs some devices under his 
experiment account. \lois(does Bob have to stipulate how many phones he needs? I don't recall seeing that factor mentioned.)

%Recall that a testbed-specific key, \path{key.sensibility}, is distributed
%with the Sensibility Testbed app downloaded and installed by device
%owners (Section~\ref{sec-owner-participate}). 
%
%\yanyan{Albert thinks this is too much detail.}
%At this moment,  Bob has obtained an account with the clearinghouse.
%and is assigned a pair of public and 
%private keys, \path{key.bob-pub} and \path{key.bob-priv}, by the
%clearinghouse. 
When Bob requests a device, and the clearinghouse
happens to find that Alice's device is available, the clearinghouse then 
%adds Bob's public key, \path{key.bob-pub}, to
%the sandbox on Alice's device. This indicates that Bob is
%authorized to use this sandbox on Alice's device, and 
assigns Alice's device to Bob's experiment account by placing Bob's
public key on Alice's device. It then instructs 
the sandbox on Alice's device to add Bob's policies by preloading
a set of blurring layer code \loisd(has the concept of blurring layer code been explained before this?). At this point, Bob can access Alice's 
device through the experiment manager, just like using \texttt{ssh}.
%Bob writes his experiment 
%code in the Python-like language supported by our secure sandbox.
%The following is a snipet of code that gets location coordinates 
%from a device:

Next, Bob uploads his code to Alice's device and 
runs his experiment. When he accesses the device through
the experiment manager, the sandbox on Alice's device 
applies the data access policies loaded by the clearinghouse: For 
policy (1), the sandbox blurs the location
information returned from Alice's phone down to the coordinates
of the nearest city; for policy (2), the sandbox blocks the
access to cell IDs; for policy (3), the sandbox limits the rate
of GPS location and cellular network queries from Bob's
experiment to one in every 10 minutes.

\subsection{Sensibility Testbed's Default Policies}\label{sec-irb-policies}

%In the domain of IRB, Alice and Bob are the participating subject, and 
%a researcher who conducts a research study on the subject, respectively.
%
%
%\textbf{Sensibility Testbed's default policies.}

\begin{table}
\scriptsize
\centering

\bgroup
\def\arraystretch{1.15}% % for table padding
\begin{tabular}{|l|c|c|c|}
\hline
\multirow{2}{*}{\bf Sensor} & 
\multicolumn{3}{c|}{\bf Default policy} \\\cline{2-4}
& {\bf LR} & {\bf MR} & {\bf HR} \\\hline

Battery (plug-in type, level, technology, etc.) & \tickmark &  & \\ \hline
Bluetooth (local name, scan mode, etc.) & & \tickmark & \\ \hline

\multirow{2}{5.5cm}{Cellular network (cell ID, area code, country code, 
operator name, etc.)} & & \multirow{2}{*}{\tickmark} & \\ 
& & & \\ \hline

Location (latitude, longitude, altitude, speed, etc.) & & \tickmark & \\ \hline
Settings (screen brightness, ringer volume, etc.) & & \tickmark & \\ \hline

\multirow{2}{5.5cm}{Motion sensors (accelerometer, 
gyroscope, magnetometer, orientation , etc.)} & & \multirow{2}{*}{\tickmark} & \\ 
& & & \\ \hline

\multirow{2}{5.5cm}{WiFi network (information about the 
currently active access point, and WiFi scan result)} & & \multirow{2}{*}{\tickmark} & \\ 
& & & \\ \hline 

%Start/stop activities & & & \xmark \\ \hline 
%Running applications & & & \xmark \\ \hline 
Camera (take pictures, record videos) & & & \xmark \\ \hline 
Intent (scan barcode, search, etc.) & & & \xmark \\ \hline 
Address book & & & \xmark \\ \hline 
Microphone (voice record) & & & \xmark \\ \hline 
SMS (send/receive messages, delete messages) & & & \xmark \\ \hline 

\end{tabular}
\egroup

\caption{\small Sensibility Testbed's default policies for sensors. LR/MR/HR
stands for low/moderate/high risk, respectively. Only sensors that have low to 
moderate risks are allowed (\tickmark). Sensors that are highly risky are 
disabled by default (\xmark).}
\label{tab:default}
%\vspace{-10pt}
\end{table}


Note that Bob cannot request complete access to all sensors
even if his IRB approves such a policy. The Sensibility Testbed's
own IRB designates a set of default policies to access sensors in a
way that is low to moderate risk, Only those sensors listed on our project 
wiki page~\cite{sensor-api} are accessible to a researcher. 
A summary of these sensors is listed in Table~\ref{tab:default}, 
where each sensor is categorized as of low, moderate or high 
privacy risk.

The list of sensors that Sensibility Testbed provides are of moderate 
to low privacy risks (marked by \tickmark), and the testbed further provides policy enforcement
(Section~\ref{sec-policy}) to protect all the sensor data. Sensors such as cameras and microphones,
that are deemed sensitive are not 
exposed to experiment code (marked by \xmark). This is motivated by the Android system, 
where permissions are categorized into different protection levels~\cite{level}:
\textit{normal} permissions are automatically granted to the apps when 
asked, \textit{dangerous} permissions are given based upon the 
user's consent, and so on. Similar to the Android permission categories, 
%we divide sensors into different risk levels, as shown 
%in Table~\ref{tab:default}. 
%Sensors with low to moderate risk are 
%allowed and protected by IRB policies. Sensors of high risk are 
%disabled by default. 
we divide sensors into different risk levels by the consequences
of a potential attack. If a microphone is controlled by 
a malicious party, it can be used to intelligently choose data of a 
higher value (e.g., credit card number, password) to record~\cite{zhang2015leave}. On the other 
hand, in order to infer credit card number or password typed on a 
smartphone using motion sensors, an sophisticated algorithm needs to 
be installed on the device and constantly learns 
the patterns of data generated by accelerometer or gyroscope.\lois(I like this expanded explanation but this particular sentence is a bit unclear) Therefore, 
compared to motion sensors, a microphone is considered of higher risk.

Although high-risk sensors are disabled, if such access  is critical to the study,
 a different IRB procedure could be followed. 
In this case, the research project has to go through the Sensibility 
Testbed's IRB, in addition to the researcher's IRB. 
\yanyan{if we think this is ok, then we provide specially
designed interface and policy?}
%Depending on the experiment description provided by the 
%researcher, the fields marked with a (*) are the ones that will be blurred.
%
%
As a result, Sensibility Testbed does not
provide unfettered access to all sensors. 
%Access to sensors of
%higher risk, e.g., the policies that request restricted sensor data, 
%or at higher frequencies than our default policies, 
%needs to go through the Sensibility Testbed's IRB,
%in addition to the researcher's IRB. 
The default policies serve as a common denominator to all 
researchers' IRB policies. In most cases, we expect
that a researcher need only go through their local IRB to get
the sensor access they need for their experiment. 

