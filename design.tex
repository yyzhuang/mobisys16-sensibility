\section{Sensibility Testbed Design}\label{sec-design}

This section describes the design of Sensibility Testbed. On one
hand, Sensibility Testbed manages how device owners make their
devices accessible to the research community. On the other hand,
it offers technical resources to researchers that allow them to
securely collect data from remote mobile devices. We start from 
an overview of the testbed infrastructure (Section~\ref{sec-overview}), 
and a description of each testbed component 
(Section~\ref{sec-component}), then finally a discussion of how all the 
components work together to assist experimenters' experimentation,
and protect device owners' personal information 
(Section~\ref{sec-scenario}).

\subsection{Overview}\label{sec-overview}

\begin{figure}
\center{\includegraphics[width=\columnwidth]{figs/arch.pdf}}
%\vspace*{-20pt}
\caption{\small Sensibility Testbed architecture. \yanyan{need 
to replot this diagram.}
\label{fig-arch}}
\end{figure}

\subsubsection{Interacting Parties}\label{sec-parties}
In Sensibility Testbed, there are three types of interacting
parties, as shown in Figure~\ref{fig-arch}: mobile devices 
owned by ordinary people, with our app installed; a 
clearinghouse server that discovers and configures
participating devices; and researchers wanting to run
experiments on mobile devices. 

Mobile devices
provide resources and data for researchers to use in their
experiments. In order to perform safe experiments on mobile
devices, a researcher must provide to the clearinghouse server
the IRB policies from his institute for accessing devices 
(Section~\ref{sec-ch}). \yanyan{need a screenshot} These 
policies restrict what and how data can be accessed by the 
researcher. The
clearinghouse server helps the researcher acquire and manage
devices, and also codifies the policies specified by the
researcher's IRB into data blurring layers that are enforced on
mobile devices (Section~\ref{sec-policy}). Such a process can protect device
owners' personal information. After obtaining remote sandboxes
and having IRB policies in place, researchers can perform
experiments on the devices using their credentials assigned by
the clearinghouse. Researchers' code runs in a sandbox on any
remote device that isolates the code from the rest of the device
host system (Section~\ref{sec-repy}). To control the execution of 
code, the researcher uses a local machine to manage the 
experiments via an experiment tool (ET). This tool can deploy 
and run experiments in sandboxes on remote devices that are 
acquired through the clearinghouse (Section~\ref{sec-emt}).

\subsubsection{Enforcing IRB Approved Policies}
Experiment code on smartphones and tablets mainly access
the \textit{sensors} on these devices. In this work, we broadly 
define sensors as the hardware components that can record 
phenomena about the physical world, such as the WiFi/cellular 
network, GPS location, movement acceleration, etc. Although
these sensors provide much convenience in everyday apps, 
they expose sensitive information that may contain device 
owner's personal information.

However, there are ways to use these sensor data without 
compromising device owners' privacy.
A recent research study has shown that more than half of the 
surveyed individuals stated no problem in supplying imprecise 
sensor data from their personal devices to protect their 
privacy~\cite{fawaz2014location}. Based on this fact, restricting 
the precision, or \textit{blurring 
data}, can be a good privacy protection mechanism we can 
provide for end users. 

In Sensibility Testbed, the data blurring 
mechanism is implemented as reference monitors in a sandbox, 
each enforcing an access control 
policy over sensor access~\cite{ref}. Using a sandboxing 
technique in our prior work~\cite{cappos2010retaining}, we can 
interject code to implement privacy policies and control what 
happens with the data gathered on the device. A sensor access 
policy can (1) reduce 
the precision of the raw sensor data returned from a device, such
as returning the device location at the city granularity; (2) restrict 
the frequency of accessing a sensor, such as the polling rate of 
accelerometer, to avoid password inferring; and (3) disable the 
access to a certain sensor in sensitive situations, such as 
turning off camera when a device is at residential or work areas.
The last policy is our ongoing work. This paper mainly handles
the first two types of policies. 

\subsection{Testbed Components}\label{sec-component}

The components of Sensibility Testbed are shown in Figure~\ref{fig-arch}.
They are central to the operation of the testbed. 
In this section, we describe the function of each component: the
secure sandbox~\cite{cappos2010retaining}, the clearinghouse~\cite{ch}, 
and an experiment management tool.

\subsubsection{Secure Sandbox}\label{sec-repy}

%In Sensibility Testbed, all experiments execute in a secure 
%sandbox on the end devices.
%Experimenters' code executes in a sandbox that isolates the 
%experiment code from the device host system. 
Instead of developing full-fledged Android apps, all 
experiments in Sensibility Testbed are written in a language
similar to Python, and run in a restricted, Python-based
sandbox called Repy, or Restricted Python\footnote{This is the 
same security-reviewed sandbox~\cite{cappos2010retaining} used in
our prior work, the Seattle testbed~\cite{seattle}. This sandbox
mitigates the impact of bugs in experimenter code. It
has been deployed on the Seattle testbed over the last 6 years.
Our experience has shown that the risk of it being faulty is
very low.}. The sandbox exposes a programming 
interface~\cite{repyv2} to experimenter programs to access 
resources on the device. It provides performance isolation: 
Repy uses OS hooks to monitor and control the amount of 
CPU and memory used by a sandboxed program. To restrict 
other resources, such as network and disk I/O, the sandbox 
interposes on system calls that use these resources and 
prevents or delays the execution of these calls if they exceed 
their configured quota. As a result, this sandbox limits 
what a sandboxed program can do. For example, reading from and writing to the file system can
only occur in a per-experiment directory; sending and receiving
data via the network interface cannot exceed a configured rate;
CPU, memory and battery consumption cannot exceed a limit, etc.
Therefore, the sandbox isolates the experiment program from 
the rest of the device. Due to this isolation, different experimenters
can have access to different sandboxes on the same device,
and each sandbox is isolated from one another.

More importantly, the sandbox allows us to interject
code to implement privacy policies and control what happens with
the data gathered on the device. For  an experiment
that involves GPS location, a privacy policy might restrict the
level of data granularity available to the experiment. For example, it can
obfuscate GPS location such that it only identifies the center
of the city that the device is located in, rather than the exact
location. Using the same technique, IP addresses may be anonymized, 
specific sensor access can be disabled entirely, and the frequency to 
access a certain sensor can be controlled. Such privacy
protection is a contribution of Sensibility Testbed, which does
not exist in any prior work. The details of policy implementation 
are in Section~\ref{sec-policy}.

To install the sandbox on a device, a device owner downloads 
our Sensibility Testbed app from the Google Play Store \cite{sensibility-app}.
The app contains the Repy sandbox, native Android code to 
start or stop the sandbox, and interface to communicate with the testbed 
infrastructure, mainly the clearinghouse (described below). 
However, the Repy sandbox does not include calls to access sensors. 
To obtain the sensor data, we need to extend the sandbox. 
The extended Repy sandbox will be described in Section~\ref{sec-repy-ext}
with more details.

\subsubsection{Clearinghouse}\label{sec-ch}
The clearinghouse~\cite{ch} is a testbed server that keeps track of 
experimenters and mediates experimenter access to the 
available devices. It allows experimenters to register 
accounts and share access to a common pool of devices.
It does so by looking up available devices, and assigning
them to researcher's experiment account upon request 
(details in Section~\ref{sec-acquire-run}). 
Its key role is to facilitate device sharing, 
which relieves individual experimenters from recruiting 
devices for their experiments.

The clearinghouse
plays an intermediate role between the experimenter and 
the device owner.
As described in Section~\ref{sec-parties}, when an 
experimenter registers at the clearinghouse, he
needs to provide his IRB policies. These policies ensure that
the researcher cannot conduct experiments to access data that
extend beyond the experiment policy. The clearinghouse 
translates and codifies each policy, and instructs the 
sandboxes on remote devices to implement these policies. 
When experiment code is running in the sandbox, the 
policies will be applied to restrict the precision of sensor 
data or the frequency to access sensors. This Sensibility Testbed
clearinghouse protocol for research plays a central role in
easing the device recruitment and experiment setup for experimenters, 
and ensures the enforcement
of privacy policy\footnote{The Sensibility Testbed Clearinghouse
protocol for research with human subjects has been approved by
the IRB at New York University. \yanyan{might need a link to
your approval letter or ref number}}. 

\subsubsection{Experiment Tool}\label{sec-emt}

To run code remotely on mobile device(s), an experimenter uses an
experiment tool (ET) from his local machine 
%which contains Bob's private key, \path{key.bob-priv}, 
to access the sandboxes on the remote device. This tool is a 
light-weight command line console and can
be downloaded from the clearinghouse~\cite{demo-kit}.
The remote access process
occurs directly between the experimenter's local machine and the 
remote device. The clearinghouse is not involved, and it does not store any
data on the experimenter's behalf. After collecting the data he needs, the
experimenter can use ET to download data from the remote devices. 
Alternatively, the experimenter can set up his own server to store all 
the data, or use a data store service we provide called Sensevis~\cite{sensevis} 
(not shown in Figure~\ref{fig-arch})\footnote{
If an experimenter stores data at his own server, he must use protective
measures to ensure that the data sent from the mobile devices is
properly encrypted, and the server storage cannot be tampered
with by any other parties. For example, the experimenter needs to register
his server by providing the server's certificate and URL to our
clearinghouse. The clearinghouse then instructs the devices
accessible to the experimenter that all the sensor data collected should be
sent to this server. The sandboxes on these devices then issue
HTTPS POST using the server's certificate, and send encrypted
data to the experimenter's server. After the data is collected, how to store
the data securely is mandated by the experimenter's IRB.}.

\smallskip
\textbf{Summary.}
%Prior to running an experiment on Sensibility Testbed, a
%experimenter first fills out a form in plain text to describe the
%purpose of the research experiment. This experiment description
%is created at the Sensibility Testbed clearinghouse
%where the researcher indicates the type of data to be collected,
%how that data will be used and stored, and so on. 
%
%Once this information is collected from the researcher, the
%clearinghouse automatically generates a set of blurring layers
%that implements the experiment policy (Section~\ref{sec-policy}). In
%Sensibility Testbed, researchers can collect data from the
%sensors on the device, such as GPS, Bluetooth, battery
%information, accelerometer, light, and orientation,
%etc. The blurring layers we provide consist of
%data access restrictions, created in accordance with
%researcher's experiment description, by using the Sensibility
%Testbed's sandboxing technique 
%(Section~\ref{sec-repy})~\cite{cappos2010retaining}. These restrictions ensure that
%the researcher cannot conduct experiments to access data that
%extend beyond the experiment policy. 
%
%This Sensibility Testbed
%clearinghouse protocol for research plays a central role in
%easing the approval process of IRB, and ensures the enforcement
%of privacy policy\footnote{The Sensibility Testbed Clearinghouse
%protocol for research with human subjects has been approved by
%the IRB at New York University. \yanyan{might need a link to
%your approval letter or ref number}}. 
Using Sensibility Testbed, experimenters are able to go 
through a streamlined process of device recruitment and 
experiment setup, and device owners' privacy is protected
from any inadvertent or malicious attempt.
The testbed reduces the burden from both the device owners and
experimenters. 

By agreeing to our general usage policy, device 
owners of different ages, from different countries, with different
background need only to opt in \textit{once} to our testbed as 
volunteers. An experimenter who wants to conduct a research 
experiment 
%requests devices through our clearinghouse, which assigns 
%them devices from a set of available resources. As a result, 
%the researcher
does not need to get consent from each subject for each individual
experiment. 
%the device owners do not need to give consents 
%multiple times, to each project of each researcher. 
In the following, we use several use cases to
demonstrate how device owners opt in as volunteers, and how
researchers do experiment on devices without compromising device
owners' privacy.

\subsection{Testbed Scenarios}\label{sec-scenario}

To demonstrate how Sensibility Testbed lowers the barriers for
researchers to perform research experiments on mobile devices,
and protects device owners' privacy, this section will go
through several scenarios. In these scenarios, a smartphone owner, Alice,
participates in the testbed; an experimenter, Bob, runs code on
Sensibility Testbed using Alice's smartphone, among other
devices. Specifically, Bob wants to know the cellular service
quality in major cities. As such, he needs location information
of individual devices, their cellular service provider, network
type (3G, 4G, LTE, etc.), and signal strength.

\subsubsection{Smartphone Owner Participates in the Testbed}
\label{sec-owner-participate}

When Alice, a smartphone owner, decides to participate in
Sensibility Testbed, she first downloads our Sensibility Testbed
app~\cite{sensibility-app}, currently supports Android devices.
This app contains a testbed key \path{key.sensibility} used for 
device discovery, the Repy sandbox (Section~\ref{sec-repy}), 
and native Android code to handle user interaction, communicate 
with the clearinghouse, and so on. Before installation, the app displays a
consent form to Alice \yanyan{cite link}, where she can review
the testbed's general usage policy \yanyan{cite link}. If Alice
agrees to the terms and policy, the app will be installed on her
device. When the app is started, Alice's device can be
discovered by the clearinghouse (Section~\ref{sec-acquire-run}). 
To keep track of Alice's device, the
clearinghouse uses a database that stores her device's unique
cryptographic key \path{key.alice} generated during
installation. This key is not associated with Alice's or her
device's identity, but only the installation on the device. If
Alice uninstalls the Sensibility app, \path{key.alice} is
deleted at the clearinghouse, which effectively \textit{unlinks}
her device from any metadata stored on the clearinghouse.
Instead of uninstalling, Alice may also choose to opt out of
individual experiments.

\subsubsection{Researcher Provides IRB
Policies}\label{sec-irb-policy}

To run code on Sensibility Testbed, experimenter Bob provides a
set of detailed IRB policies by his institution. The policies specify 
what data can be accessed by a research experiment, at which 
granularity or frequency can such data be accessed, how data 
should be securely stored, and so on.
For example, Bob's experiment can (1) read location information
from devices at the granularity of a city, (2) read accurate
cellular signal strength and network type, but no information
about cell IDs should be accessed, and (3) get location and
cellular network data updates every 10 minutes. 
%Bob submits an
%experiment description for these requirements, which the
%clearinghouse will codify into policies that are later enforced
%on remote mobile devices (Section~\ref{sec-ch}).

Note that Bob cannot request access to all sensors at any rate
even his IRB approves such a policy. The Sensibility Testbed's
IRB allows a set of default policies to access to sensors in a
way that is low risk, \yanyan{cite}whose access can be pre-approved with the
researcher's local IRB. However, Sensibility Testbed does not
provide unfettered access to all sensors. Access to sensors of
higher risk needs to go through the Sensibility Testbed's IRB,
in addition to the researcher's IRB. For most cases, we expect
that a researcher need only go through their local IRB to get
the sensor access they need for their experiment. After
providing his IRB policies, Bob next can obtain an experiment
account and request a number of devices from our clearinghouse.

\subsubsection{Researcher Acquires Device(s) and Runs an
Experiment}\label{sec-acquire-run}

%The above clearinghouse protocol ensures the enforcement of data
%access policies. Additionally, 
To perform an experiment, the clearinghouse first needs to 
acquire and assign devices to Bob's experiment account. 
Recall that a testbed-specific key, \path{key.sensibility}, is distributed
with the Sensibility Testbed app downloaded and installed by device
owners (Section~\ref{sec-owner-participate}). When the app is started, 
Alice's device contacts a lookup service to advertise itself to Sensibility 
Testbed. \yanyan{add lookup service in Fig 1?}
The clearinghouse periodically queries the lookup service to
discover any new devices. Once Alice's device is discovered, the
clearinghouse obtains its install key \path{key.alice} generated
during installation, and stores this key in a database. 

%\yanyan{Albert thinks this is too much detail.}
At this moment,  Bob has obtained an account with the clearinghouse 
(Section~\ref{sec-irb-policy}).
%and is assigned a pair of public and 
%private keys, \path{key.bob-pub} and \path{key.bob-priv}, by the
%clearinghouse. 
When Bob requests a device, and the clearinghouse
happens to find that Alice's device is available, the
clearinghouse first 
%adds Bob's public key, \path{key.bob-pub}, to
%the sandbox on Alice's device. This indicates that Bob is
%authorized to use this sandbox on Alice's device, and 
assigns Alice's device to Bob's experiment account, then instructs 
the sandbox on Alice's device to add Bob's policies by preloading
a set of blurring layer code.
%Bob writes his experiment 
%code in the Python-like language supported by our secure sandbox.
%The following is a snipet of code that gets location coordinates 
%from a device:

Next, Bob needs to upload his code to Alice's devices and 
run the experiment. When he accesses the device through
the experiment management tool, the sandbox on Alice's device 
applies the data access policies loaded by the clearinghouse: For policy (1) in
Section~\ref{sec-irb-policy}, the sandbox blurs the location
information returned from Alice's phone down to the coordinates
of the nearest city; for policy (2), the sandbox blocks the
access to cell IDs; for policy (3), the sandbox limits the rate
of GPS location and cellular network queries from Bob's
experiment to one every 10 minutes.

