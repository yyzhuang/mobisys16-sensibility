\section{Evaluation}

In this section we evaluate the effectiveness of Sensibility Testbed's 
privacy mechanisms, its usability as a mobile testbed, and its 
performance.

\todo{Q: how well the testbed protects privacy, and still allows experiments
to function? A: \ref{sec-accurate} and \ref{sec-function}.}

\todo{Q: how people feel if their privacy has been protected? are they willing
to participate? A: \ref{sec-usability}}

\todo{Q: how is the testbed being used? A: deployment experience: 
number of people, projects, issues, hackathon. (\ref{sec-external})}

\todo{Q: how easy is it to use the testbed? A: ease of use: 
compared to common measurement app (Android), 
how many lines of code can one save (\ref{sec-ease})}

\todo{Q: what's the overhead? A: overhead and battery (\ref{sec-benchmark})}


\subsection{Privacy Protection with IRB Policies}

\subsubsection{Data Accuracy}\label{sec-accurate}

\yanyan{compare the results of an algorithm on blurred data 
collected from the testbed vs more frequent collection. this is 
useful for showing that we collect a representative sample of 
the data (Seth's algorithm?)}

\subsubsection{Experiment Functionality}\label{sec-function}

\yanyan{with the privacy protection in place, is the data we 
provide sufficient for experiments to function?}

\todo{need to find a good use case.}


\subsection{Usability}\label{sec-usability}

\yanyan{user survey on privacy: show how people feel if their 
privacy has been protected, whether device owners feel the 
protection is enough.}

\yanyan{incentives to participants}

\subsection{Deployment Experience}\label{sec-deployment}


\subsubsection{External Projects and Collaboration}\label{sec-external}

Sensibility Testbed has been adopted by projects such as 
Open3G~\cite{open3g} that investigates on cellular technologies 
and coverage, a vehicle data collection project~\cite{reininger2015first} 
that monitors vehicle traffic patterns. The experience from the 
project developers was positive, and they identified cases where our 
instructions for use were unclear. Despite these clarification issues, in 
the case of Open3G, the developer reported that writing the code to
get cellular information took about an hour and they have used our
testbed since 2012. We are currently in discussions with other 
researchers about integrating Sensibility Testbed into their research
projects, such as monitoring construction safety.

In 2014, we hosted a Hack-a-thon styled, one-day workshop co-located with 
the IEEE Sensors Applications Symposium (SAS) \cite{sas}. About twenty 
workshop participants with various backgrounds worked in teams 
for six hours and built four functional applications using Sensibility 
Testbed. None of the participants had any prior experience with 
this testbed, and many of them had no background in Computer
Science. With this success, we hosted a second workshop with 
SAS in 2015, and will continue in 2016. Interesting experiments 
developed by the participants include a device network monitor: 
when the battery level of a device is low, WiFi or Bluetooth that 
requires high power is turned off.

\subsubsection{Ease of Use}\label{sec-ease}

\yanyan{how much time a student spent doing some
measurements (like Thomas's motion detection alg).}

\yanyan{compare to other projects,  developers can save
thousands of lines of code.}

\subsection{Microbenchmarks}\label{sec-benchmark}

\yanyan{briefly show how effective/efficient it collects data, 
and overhead compared to native.}

\subsubsection{Measurement Overhead}

\subsubsection{Battery Consumption}
