\section{Evaluation}\label{sec-eval}

In this section we evaluate the effectiveness of Sensibility Testbed's 
privacy mechanisms, its usability as a mobile testbed, and its 
performance. To investigate whether the policies provided by 
Sensibility Testbed are representative, we surveyed the projects
in the past few years and identified their security and privacy 
mechanisms. We analyzed if Sensibility Testbed can provide policies 
to prevent common privacy attacks in Section~\ref{sec-our-policies}. 
We then show an example by applying Sensibility Testbed's 
policies to an experiment in Section~\ref{sec-experiment}. We 
demonstrate that with the policies in place, we can prevent device
fingerprinting using accelerometers. We show the performance overhead of 
Sensibility Testbed in Section~\ref{sec-benchmark}, and our 
experience in Section~\ref{sec-deployment}.

\subsection{Sensibility Testbed's Privacy Policies}\label{sec-our-policies}

The policies in \sysname are implemented as blurring layers, as described
in Section~\ref{sec-bob-policy}. Although the number of policies can be
infinite, similar policies can reuse common code, and be parameterized 
given researcher supplied specifications. 

In this section, we investigate which policies can be set by \sysname to 
prevent common attacks, and how \sysname's privacy protection compared 
to other general defense frameworks.


\begin{table}
\scriptsize
\centering

\bgroup
\def\arraystretch{1.15}% % for table padding
\begin{tabular}{|l|l|l|}
\hline
{\bf Project} & {\bf Sensor} & {\bf Sensibility Testbed policy}  \\\hline

\multirow{2}{*}{EnCore~\cite{aditya2014encore}}  & \multirow{2}{*}{Bluetooth} & 
\multirow{2}{3.5cm}{Randomize Bluetooth MAC addresses.}  \\
& &  \\\hline
% attack: associating Bluetooth MAC address with device
% solution: use Bluetooth 4.0's native support for randomized MAC address

%\multirow{2}{*}{\cite{chen2014sensor}\textsuperscript{*}} & Accelerometer 
%& \tickmark &   &  \\ \cline{2-5}
%& Camera & & \tickmark & \\ \hline
%% attack: 2D photo/video attack (use other's photo/video as input to facial recognition) 
%% and virtual camera attack (using a virtual cam to modify videos)
%% solution: (1) use accelerometer to infer the orientation of front camera, 
%% (2) user shaking the phone, and separate the data of shaking video and
%% shaking accelerometer to ensure a real 3D face

\multirow{2}{*}{\cite{bojinov2014mobile}\textsuperscript{*}} & Microphone  
& Add random (small) noises. \\ \cline{2-3}
& Accelerometer & Random (small) rotation\textsuperscript{\dag}.  \\ \hline
%& Gyroscope & \tickmark  \\ \cline{2-3}
%& Magnetometer & \tickmark  \\ \cline{2-3}
%& Ambient light & \tickmark  \\ \cline{2-3}
%& Touch screen &  \\ \hline
% attack: use microphone and accelerometer to fingerprint devices
% solution: this is a fingerprinting paper. it measures sensor's 
% imperfection. though not evaluated, 
% they showed in Tab 1 that gyroscope, magnetometer, etc. all 
% have biases and imperfection that can be used for fingerprinting.

AccelPrint~\cite{dey2014accelprint} & Accelerometer & Random (small) 
rotation.   \\ \hline
% attack: use accelerometer to fingerprint devices
% solution: this is a fingerprinting paper. one can fingerprint an accelerometer
% when it is vibrating (eg, during an incoming call) and measure its imperfection

\cite{clarkson2012breaking}\textsuperscript{*} & Microphone & 
Add random (small) noises. \\ \hline

\cite{lukas2006digital}\textsuperscript{*} & Camera & Limit pixel resolution. \\ \hline
%& Location (GPS) & \tickmark  \\ \hline


%Koi~\cite{guha2012koi} & Location & \tickmark  \\ \hline
% attack: location tracking
% solution: apps set location based triggers, and have them fired
% based on location updates. eg, instead of location lookup, it does
% location matching: when the phone is in a grocery store, it notifies
% the app that set this locaiton trigger
% prevents linkage between location and user identity

Cach{\'e}~\cite{amini2011cache} & Location & Approximate to a zipcode region/city.    \\\hline
% attack: location privacy
% solution: caching locaiton content in advance. retrieve content locally
% when it is needed. this is like approximate location with a zipcode/city


\multirow{4}{*}{FindingMiMo \cite{shin2011findingmimo}} 
& WiFi & \multirow{2}{3cm}{Access frequency $\leq$ once per 2 minutes.}  \\ \cline{2-2}
& Location &  \\\cline{2-3}
& Accelerometer & \multirow{2}{*}{Unspecified.}  \\ \cline{2-2}
& Magnetometer &   \\ \hline
% attack: this is a tracker for device. it tracks and locates a missing mobile device
% in indoor environments by observing the ambient features such as radio signals, 
% and retrieving location information from ambient observations
% sec 5.1:  The sampling interval for GPS sensing is
%2 minutes; we activates it for 30 seconds for single positioning.
%The interval of Wi-Fi scanning is adaptively determined, based on
%a user’s movement. In the move state, the system stores Wi-Fi
%vectors every 5 seconds. In the stationary state, we activates Wi-Fi 
%module for 30 seconds to scan the surrounding APs every 10
%seconds. The sampling interval is set at 2 minutes


\multirow{2}{*}{Cloaking~\cite{gruteser2003anonymous}} & \multirow{2}{*}{Location} 
& \multirow{2}{3.5cm}{Radomized location within a 100 meter radius.}  \\ 
& & \\ \hline
% attack: location tracking
% solution: adjust the resolution of location information along spatial or 
% temperal dimensions to meet the specified anonymity constraints

Locaccino~\cite{toch2010empirical} & Location & Approximate to a zipcode region/city.  \\ \hline
% attack: location tracking
% solution: survey showed that users are more comfortable sharing their
% location when they are at places visited by a large and diverse set of people

\multirow{3}{*}{Accomplice~\cite{han2012accomplice}} & \multirow{2}{*}{Location} 
& \multirow{2}{3.5cm}{Radomized location within a 200 meter radius.}  \\ 
& & \\ \cline{2-3}
& Accelerometer & Unspecified.  \\ \hline
% attack: location tracking using accelerometer
% solution: this is an attack paper. accelerometer can be used to track location
% sec IV.B: 200 m radius

\cite{shokri2011quantifying}\textsuperscript{*}
& Location & Approximate to a zipcode region/city.  \\ \hline
% attack: tracking and localization attacks
% solution: a generic theoretical framework for modeling and evaluating
% location privacy, lots of stats. tracking and localization attacks are 
% modeled as statistical inferernce problems, and metric to evaluate 
% location privacy is an adversary's expected error.

\multirow{3}{*}{Where's Wally \cite{polakis2015s}}%\textsuperscript{*}}
& \multirow{3}{*}{Location} & \multirow{3}{3.5cm}{Radomized location 
in an area of 700 m$^2$, access frequency $\leq$ 500 times per hour.}  \\
&& \\ %\cline{2-2}
&& \\ \hline
% attack: tracking and localization attacks
% solution: many apps use location proximity (notify who is nearby, at 
% what distance). the paper theoretically model the problem of locating
% users as a search problem, and device optimal algorithm to realize
% those attacks

\multirow{3}{*}{AnonySense~\cite{kapadia2008anonysense}} 
& \multirow{2}{*}{Location} & \multirow{2}{3.5cm}{Approximate to a small region (size unspecified).}  \\ 
& & \\ \cline{2-3}
& \multirow{2}{1.8cm}{Network interface addresses} & 
\multirow{2}{*}{Randomize IP and MAC addresses.}  \\
& & \\ \hline
% attack: localization attacks
% solution: partition the area of the user into tessellation, and report 
% the tile number instead of real location of the user. also changes
% IP and MAC addresses

\cite{andres2013geo}\textsuperscript{*} & Location & Radomized 
location within radius r.   \\ \hline

\cite{bordenabe2014optimal}\textsuperscript{*}
& Location & Radomized location within a grid.   \\ \hline
% attack: localization attacks
% solution: optimizing the trade-off between geo-indistinguishability 
% and quality of service. 

%PrivStats~\cite{popa2011privacy} & Location & \tickmark   \\ \hline
% attack: localization attacks
% solution: compute aggregate statistics over location data. achieves 
% both strong guarantees of location privacy and protection against 
% cheating clients.

LP-Guardian~\cite{fawaz2014location} 
& Location & \tickmark   \\ \hline
% attack: tracking, profiling, and identification threats
% solution:  anonymize location in the background and let user to choose 
% appropriate anonymization strategy for foreground location access; 
% feed app with the location granularity necessary to serve the user; 
% for apps that require constant monitoring, replace the real location
% with dummy ones that belong to a synthetic route

ACCessory~\cite{owusu2012accessory} & Accelerometer & 
Access frequency $\leq$ 100 Hz.  \\ \hline
% attack: accelerometer can be extracted to monitor inputs
% solution: this is an attack paper, showing it's possible to extract entered 
% text. in Sec 4.1/Fig 5 they mentioned sensor rate@100Hz significantly
% improved accuracy.

\multirow{2}{*}{TapPrints~\cite{miluzzo2012tapprints}} & Accelerometer 
& \multirow{2}{*}{Access frequency $\leq$ 100 Hz.}   \\ \cline{2-2}
& Gyroscope &  \\ \hline
% attack: monitor user inputs like keyboard and icons
% solution: this is an attack paper, showing it's possible to infer location
% of taps using motion sensor + machine learning
% in Sec 4 they mentioned they had to use sensor rate@100Hz

\multirow{2}{*}{TapLogger~\cite{xu2012taplogger}} & Accelerometer & \multirow{2}{*}{Access 
frequency $\leq$ 25 Hz.}   \\ \cline{2-2}
& Gyroscope &   \\ \hline
% attack: accelerometer and gyroscope can be extracted to monitor inputs
% solution: this is an attack paper, showing it's possible to infer 
% input using accelerometer and gyroscope. Tab 1 in the paper shows
% "useful" sensor rate is: accelerometer/orientation both 25-50Hz, .

TouchLogger~\cite{cai2011touchlogger} & Gyroscope & Access 
frequency $\leq$ 33 Hz. \\ \hline
% attack: gyroscope can be extracted to monitor inputs
% solution: this is an attack paper, showing it's possible to infer 
% input using gyroscope

\cite{aviv2012practicality}\textsuperscript{*} & Accelerometer & Access 
frequency $\leq$ 76 Hz.  \\ \hline
% attack: accelerometer can be extracted to monitor inputs
% solution: another attack paper. their method is sample-rate independent. 
% in Tab 1 they used rates from 25 to 62Hz. on page 8, "all devices perform well
% above random guessing, suggesting that the features are reasonably
% resilient to sample rate fluctuations, as addressed by the samplenormalized
% features (see Section 6)"

\multirow{2}{*}{\cite{cai2012practicality}\textsuperscript{*}} & Accelerometer 
& \multirow{2}{*}{Access frequency $\leq$ 30 Hz.}  \\ \cline{2-2}
& Gyroscope &  \\ \hline
% attack: accelerometer and gyroscope can be extracted to monitor inputs
% solution: another attack paper. they showed attack remains effective 
% even though the accuracy is affected by user habits, device dimension, 
% screen orientation, and keyboard layout. 

\multirow{2}{*}{\cite{liu2015good}\textsuperscript{*}}
& Accelerometer & \tickmark   \\ \cline{2-3}
& Microphone  &  \\ \hline 
% attack: use accelerometer on smart watches to monitor inputs (more
% challenging than accelerometer on smartphones)
% solution: this is an attack paper. microphone is used to help 
% accelerometer during the detection

Gyrophone~\cite{michalevsky2014gyrophone} & Gyroscope 
& Access frequency $\leq$ 100 Hz.  \\ \hline
% attack: use gyroscope to record voice
% solution: this is an attack paper

\cite{jiang2012isolating}\textsuperscript{*} & Cellular & Disable dialing phone numbers.   \\\hline
% attack: voice related fraud (dial a number without the user knowing it)
% solution: detect voice-related fraud activities using call records, 
% use voice calls from domestic callers to foreign recipients and 
% a Markov Clustering based method for isolating dominant fraud 
% activities from international calls

\multicolumn{3}{l}{\textsuperscript{*}\scriptsize These projects do not have a project name.} \\ 

\multicolumn{3}{l}{\textsuperscript{\dag}\scriptsize An example of this is shown in Section~\ref{sec-experiment}.} \\ 

\end{tabular}
\egroup

\caption{\small Sensibility Testbed prevent against known attacks. 
%Default policies are listed in Table~\ref{tab:default}.
%is supported by Sensibility Testbed without extra effort. A specialized policy can 
%be supported by extending default policies. A policy is N/A if it is not possible 
%to provide support.\yanyan{specific attacks}}
}
\label{tab:policy}
%\vspace{-10pt}
\end{table}

\subsubsection{Policies Against Known Attacks}
In order to evaluate if \sysname's policies are effective in preventing common 
attacks, we surveyed 25 recent projects that 
protect against sensor related privacy issues. We examined the sensor 
privacy attacks in these projects, and how these projects control the access 
to these sensors (in terms of data granularity, if available). We show these access control 
in the form of equivalent \sysname default policies. The results are shown in 
Table~\ref{tab:policy}. In some cases, the details about data access granularity
is insufficient. We mark these cases as "unspecified". 

As shown in the table, the most common risks are identifying a device or its 
owner, locating a device, and inferring keys strokes typed by a device owner. 
For example, to prevent identifying a device, EnCore~\cite{aditya2014encore} 
uses randomized Bluetooth MAC address to prevent device tracking. This 
identification attack can also be
prevented by a Bluetooth policy in \sysname that generates a random string to 
substitute the Bluetooth MAC address. Another project FindingMiMo~\cite{shin2011findingmimo} 
locates a device by monitoring the device's location and ambient radio signal 
(particularly WiFi). It generates an ambient environment fingerprint that a 
device encouter during a day so that the device can be located. However, as 
shown in~\cite{shin2011findingmimo}, the GPS location and WiFi scan's sampling 
interval has to be less than 2 minutes. Therefore, using a \sysname policy to 
restrict the access frequency to these sensors to below once per 2 minutes can 
effectively prevent the environment fingerprinting, even if the accuracy of GPS 
and WiFi data is not reduced. When GPS signal is not available (e.g., indoors), 
FindingMiMo uses an accelerometer and magnetometer to model human steps,
but there is insufficient details about the data granularity of these two sensors. 
ACCessory~\cite{owusu2012accessory} shows that it is possible to infer key 
strokes using an accelerometer, and the key inference accuracy increases 
significantly when the sensor readings achieves 100Hz. Therefore, using a 
policy to limit accelerometer access frequency to below keyloggers' desired 
rate can make them much less effective. \yanyan{not all keyloggers require
100Hz though.}

As stated in Section~\ref{sec-policy-design}, \sysname's default policies are 
set to appropriate levels to protect against known attacks today. 
Although \sysname can set default policies for the privacy attacks, these 
levels will need to be revised over time when new attacks become available. 
\sysname's IRB allows adjusting these specific sensor access restrictions, 
and thus making the default policies stronger over time, and preventing new 
attacks with future use.

%These projects, listed in Table~\ref{tab:policy} and \ref{tab:policy-continued},
%range from social network applications~\cite{aditya2014encore} to facial
%recognition algorithms~\cite{chen2014sensor}. If their privacy need or privacy
%protection can be supported by equivalent \textit{default} policies in Sensibility 
%Testbed, a \tickmark\ is marked in the default column. These sensors belong to 
%low to moderate risk, as defined in Table~\ref{tab:default}. Similarly, if the
%sensors are of high risk, a \tickmark\ is marked in the specialized column. 
%This means that a different IRB procedure could be followed to extend 
%default policies (Section~\ref{sec-motivation}). If a sensor cannot be 
%supported by a default or specialized policy, a \xmark\ is marked in the 
%N/A column. 
%
%Out of these projects, 
%19 of them proposed privacy protection about location information, 16 of 
%them considered motion sensors such as accelerometer and gyroscope 
%risky, and 9 had concerns about wireless network such as WiFi and Bluetooth
%(connection/pairing history, MAC addresses, etc.). As expected, location (mainly GPS)
%and motion sensors (mainly accelerometer) have the most privacy and security 
%concerns~\cite{chakraborty2014ipshield}.


%\textbf{Are Sensibility Testbed's policies sufficient?}
%As shown in Table~\ref{tab:policy} and \ref{tab:policy-continued}, nearly 86\% 
%of the security and privacy issues in prior projects can be addressed in 
%Sensibility Testbed using default policies, and 13\% issues can be resolved by extending 
%default policies (specialized policies). Only about 1\% of them cannot
%be protected by the policies in Sensibility Testbed, \yanyan{I'm not sure 
%yet how to explain this. I don't know what experiment would need this.}

\subsubsection{Sensibility Testbed's Protection Compared to Other General
Defense Frameworks}
To evaluate whether \sysname's privacy protection can be as effective as other 
general defense frameworks, we surveyed another 7 recent projects that provide
protection against privacy attacks. These projects are different from the ones
in Table~\ref{tab:policy} because \yanyan{complete this.} Similar to the projects 
Table~\ref{tab:policy}, we also examined the sensor privacy attacks in these 7 
projects, and how these projects control the access to these sensors. 
As shown in Table~\ref{tab:policy-continued}, if a privacy policy can be supported
by \sysname's default policy (e.g., providing a hashed device ID), a \tickmark\ 
is placed in the default column. If a default policy is too restrictive, but an 
extended policy with \sysname's IRB can permite (e.g., access to a camera
with limited pixel resolution), then a \tickmark\ is placed in the extended column.
For example, although \sysname does not provide  information-flow tracking as 
in TaintDroid~\cite{enck2014taintdroid}, it provides equivalent restrictions to 
location, accelerometer, microphone, camera, address book, SMS messages, 
device identifiers, and network interfaces. 

As shown in the table, most of the privacy policies (x\%) can be supported by
\sysname's default policies, and y\% needs to go through extended IRB in order
to get access to sensor data. 


\begin{comment}
As another example, prior work shows 
that Android users' touch inputs can be revealed through a few attack 
techniques like keyloggers and fingerprinting. 
%For example, a smartphone's accelerometer 
%and gyroscope can disclose shift and rotation data when a user types 
%through a software keyboard. Imprecisions in sensor calibration can 
%result in a device-specific scaling and thus can be a reliable fingerprint 
%of the device. 
User generated data thus can be informative enough 
for malware to infer the key the user enters~\cite{cai2011touchlogger, 
owusu2012accessory}, or to fingerprint and identify individual 
devices~\cite{bojinov2014mobile, dey2014accelprint}. Because these 
motion sensors are accessible 
%through JavaScript in a mobile web browser, 
without requesting any 
permissions or notifying the device owner, these attack techniques are much 
less detectable. However, the chance for these techniques to succeed
depends on their sampling rate. 
%Consider an Android 
%user's average typing speed of 3 keys per second, when the sampling 
%rate goes down to once per second, the best the adversary can do is 
%just to identify 1 of these 3 keys. 
Therefore, the policies to restrict the access rates to motion sensors 
are effective when the allowed rate is lower than the best keylogger 
would require to identify a key, or the best tracker to fingerprint a device.
Section~\ref{sec-experiment} shows such an example. 
\end{comment}


%\todo{Q1.2: does ST address these concerns? A1.2: show
%  a handful of past experiments with a 90/10 rule -- 90\% of experiments
%  can be doone with virtually no mods, and the other 8\% we can write 
%  specialized policy for and 2\% it's too hard.}

\subsection{An Example Privacy Protection}\label{sec-experiment}

%\todo{Q2: is ST useful for answering research questions? A:
%compare the results of an algorithm with varying levels of 
%data precision (Seth's algorithm?), and show a diagram suggested
%by Justin.}
This section demonstrates an example policy enforcement using the 
privacy protection in Section~\ref{sec-our-policies}. In this example, an 
adversary uses an accelerometer to fingerprint a device. 
%The technique relies on the fact that imprecisions in sensor calibration can 
%result in a device-specific scaling and thus can be a reliable fingerprint of the device. 


Research has shown that using sensors on a modern smartphone, such as 
a speaker~\cite{das2014poster}, microphone~\cite{clarkson2012breaking} 
or accelerometer~\cite{bojinov2014mobile, dey2014accelprint}, one can 
build a robust device fingerprint that is independent of the software state
and survives a hard reset. The more recent accelerometer based fingerprinting
is particular interesting because an accelerometer is accessible in mobile apps
without requesting any permissions from the device owner, therefore these 
attack techniques are much less detectable. Specifically, the 
technique in~\cite{bojinov2014mobile} is based on the device-specific errors in 
accelerometer calibration, which result in unique scaling and translation of 
the measured values. In this section, we show that by using data blurring in 
\sysname, it is possible to prevent such fingerprinting. 
%while producing data that is still sufficient for benign experiments to function. 


\begin{table}
\scriptsize
\centering

\bgroup
\def\arraystretch{1.15}% % for table padding
\begin{tabular}{|l|l|c|c|c|}
\hline
\multirow{2}{.8cm}{\bf Project} & \multirow{2}{*}{\bf Sensor} & 
\multicolumn{3}{c|}{\bf Sensibility Testbed policy} \\\cline{3-5}
& & {\bf Default} & {\bf Extended} & {\bf No support} \\\hline

\multirow{8}{*}{TaintDroid~\cite{enck2014taintdroid}} & Location & \tickmark &   &  \\ \cline{2-5}
& Accelerometer & \tickmark &   &  \\ \cline{2-5}
& Microphone & & \tickmark & \\ \cline{2-5}
& Camera & & \tickmark & \\ \cline{2-5}
& Address book & & \tickmark & \\ \cline{2-5}
& SMS messages & & \tickmark & \\ \cline{2-5}
& Device IDs & \tickmark & & \\ \cline{2-5}
& Network interface\textsuperscript{\dag} & \tickmark & & \\ \hline
% attack: general privacy attacks
% solution: labels (taints) data from privacy-sensitive sources and
% transitively applies labels as sensitive data propagates
% through program variables, files, and interprocess messages.
% (maybe this is not the best example)

\multirow{5}{.8cm}{ProtectMyPrivacy \cite{agarwal2013protectmyprivacy}} & Device ID & \tickmark &  & \\ \cline{2-5}
& WiFi & \tickmark &   &  \\ \cline{2-5}
& Bluetooth & \tickmark &   & \\ \cline{2-5}
& Address book & & \tickmark & \\ \cline{2-5}
& Location & \tickmark &   &   \\\hline
% attack: general privacy issues
% solution: substitute anonymized data in place by crowd sourced recommendations

\multirow{4}{*}{CQue~\cite{parate2013leveraging}}  & Location & \tickmark &  & \\\cline{2-5}
& Accelerometer & \tickmark &   &  \\ \cline{2-5}
& Gyroscope & \tickmark &   &  \\ \cline{2-5}
& Bluetooth & \tickmark &   &   \\\hline
% attack: this paper tries to infer individual behavior by fusing a bunch of sensors 
% and provide insights into privacy leakage (use issue what-if queries to explore
% the leakage if certain sensor data is revealed to apps)
% solution: detection and fusion of sensors (math)

\multirow{3}{*}{Jigsaw~\cite{lu2010jigsaw}} & Accelerometer 
& \tickmark &   &  \\ \cline{2-5}  
& Microphone  & & \tickmark & \\ \cline{2-5}
& Location & \tickmark &   &   \\\hline
% attack: robustness to position change and energy drain
% solution: throttles the depth and sophistication of sensing when data is low quality

\multirow{6}{*}{MockDroid~\cite{beresford2011mockdroid}} 
& Location & \tickmark &  & \\\cline{2-5}
& Internet\textsuperscript{\dag} & \tickmark & & \\ \cline{2-5}
& Cellular & \tickmark &   &  \\ \cline{2-5}
& Address book & & \tickmark & \\ \cline{2-5}
& Device ID & \tickmark & & \\ \cline{2-5}
& Broadcast intent &  & \tickmark  &  \\ \hline
% attack: general privacy attacks
% solution: modify the Android OS to allow users to mock or revoke
% access to data during run time (simulating lack of location info, wireless
% network, fake device ID, never sending broadcast intent, etc.)

\multirow{6}{*}{Guardian \cite{zhang2015leave}} 
& Bluetooth & \tickmark &   & \\ \cline{2-5}
& Internet\textsuperscript{\dag} & \tickmark & & \\ \cline{2-5}
& Microphone  & & \tickmark & \\ \cline{2-5}
& Cellular & \tickmark &   &  \\ \cline{2-5}
& Motion sensors & \tickmark &   &  \\ \cline{2-5}
& CPU usage\textsuperscript{\ddag} & \tickmark & & \\\hline
% attack: rumtime information gathering (a malicious app recording
% phone conversations, etc.)
% solution: pause suspicious background apps (that records the phone 
% conversation) when the target app is running in the foreground. the 
% background apps are identified by their behaviors inferred from their
% thread names, CPU scheduling, kernel time etc. the key is when to 
% suspend and resume the suspicious background apps


\multirow{5}{*}{ipShield~\cite{chakraborty2014ipshield}} 
& Location (GPS) & \tickmark &   &  \\ \cline{2-5}
& Accelerometer & \tickmark &   &  \\ \cline{2-5}
& Gyroscope & \tickmark & &  \\ \cline{2-5}
& WiFi & \tickmark &   &  \\ \cline{2-5}
& Cellular & \tickmark &   & \\ \hline
% attack: general privacy attacks
% solution: allow users to specify privacy preferences, auto-generates
% privacy actions on sensors (supression, allow, add noise, etc). 

\multirow{3}{*}{MaskIt~\cite{gotz2012maskit}} & Location & \tickmark &   & \\\cline{2-5}
& Cellular & \tickmark &   & \\ \cline{2-5}
& Bluetooth &  \tickmark &   & \\\hline
% attack: temperal correlations (heading towards hospital means the person will
% be at the hospital)
% solution: filter a user context stream that provably preserves privacy (a privacy 
% check deciding whether to "release or suppress" the current user context). privacy 
% is defined with respect to a set of sensitive contexts specified by the user
% sec 7.1 described data (location, cellular, bluetooth)

%\multirow{2}{*}{\bf Total} & \multirow{2}{*}{\bf 77} & \multirow{2}{1cm}{\bf 
%65/77 (84.42\%)} & \multirow{2}{1cm}{\bf 11/77 (14.29\%)} & 
%\multirow{2}{1cm}{\bf 1/76 (1.29\%)} \\ & & & & \\\hline

\multicolumn{5}{l}{\textsuperscript{\dag}\scriptsize Internet connectivity policies
can be implemented by interposing socket calls.} \\

\multicolumn{5}{l}{\textsuperscript{\ddag}\scriptsize CPU usage can be obtained
through reading the files in \path{/proc/stat}.} \\

\end{tabular}
\egroup

\caption{\small \sysname's privacy protection compared to other general
defense frameworks.}
\label{tab:policy-continued}
%\vspace{-10pt}
\end{table}



\begin{figure*}
\centering
\begin{subfigure}[b]{.25\textwidth}
  \centering
  \includegraphics[width=\textwidth, trim={.9cm .5cm 1.8cm 1.3cm},clip]{figs/sz-oz.pdf}
  \caption{Data without blurring.}
  \label{fig:sub1}
\end{subfigure}%
\begin{subfigure}[b]{.25\textwidth}
  \centering
  \includegraphics[width=\textwidth, trim={.9cm .5cm 1.8cm 1.3cm},clip]{figs/sz-oz-rand10.pdf}
  \caption{$[0, 10^{\circ}]$ rotation.}
  \label{fig:sub2}
\end{subfigure}%
\begin{subfigure}[b]{.25\textwidth}
  \centering
  \includegraphics[width=\textwidth, trim={.9cm .5cm 1.8cm 1.3cm},clip]{figs/sz-oz-rand20.pdf}
  \caption{$[0, 20^{\circ}]$ rotation.}
  \label{fig:sub3}
\end{subfigure}%
\begin{subfigure}[b]{.25\textwidth}
  \centering
  \includegraphics[width=\textwidth, trim={.9cm .5cm 1.8cm 1.3cm},clip]{figs/sz-oz-rand30.pdf}
  \caption{$[0, 30^{\circ}]$ rotation.}
  \label{fig:sub3}
\end{subfigure}%

\caption{\small Accelerometer fingerprinting without blurring, and with different levels of 
blurring that partially randomize the accelerometer data. Each device's $S_z$-$O_z$
pair is represented by the median and standard deviation of $10,000$ data samples.
\yanyan{it's better to draw a scatter plot with the result of a clustering alg overlaid on it.}}

\label{fig:fingerprinting}
\end{figure*}

\textbf{Background on accelerometer calibration.}
An accelerometer measures the acceleration force to a device along three 
physical axes. However, during the manufacturing and assmebly process, 
imperfections or variations are introduced for each individual hardware 
instance, resulting in biases in the sampled data that are unique to a specific
accelerometer~\cite{doscher1998accelerometer}. The value of a measurement
by an accelerometer, denoted by $v_m$, can be approximated as a linear 
funciton of its true value, denoted by $v_t$, as $v_m=v_tS+O$. $S$ and $O$
are the sensitivity and offset of the accelerometer, respectively. If an 
accelerometer has no calibration bias, then $S=1$ and $O=0$. Note that
this linear bias is an approximation of the real bias. In practice, $S$ and $O$
manifest randomness.

\textbf{Estimating biases along one axis.}
Like in~\cite{bojinov2014mobile}, we take advantage of the Earth's gravity
$g$, and analyze the biases along the $Z$-axis, $S_z$ and $O_z$. To 
estimate two unknows, $S_z$ and $O_z$, we need two measurements: an
accelerometer's measurement value along the $Z$-axis when the device is 
facing up ($z_m^+$) and facing down ($z_m^-$). Therefore, we have

\begin{equation}\label{eq:fxy}
\left\{
    \begin{array}{c}
      S_z = (z_m^+ - z_m^-)/2g \\
      O_z = (z_m^+ + z_m^-)/2
    \end{array}
  \right..
\end{equation}

The authors of~\cite{bojinov2014mobile} showed that this method 
produces very satisfactory results in a range of settings even 
if the surface on which the phone lies is not perfectly level.

\textbf{Data blurring and fingerprinting.}
We carried out an accelerometer fingerprinting experiment on a group
of 10 Android devices, with $10,000$ measurement samples from each device.
Figure~\ref{fig:fingerprinting}(a) shows the estimated $S_z$ and $O_z$ in Eq.~(\ref{eq:fxy})
based on unfiltered accelerometer data, where each device is represented 
by the medians and variances of $S_z$ and $O_z$ along the X and Y axis. 
The fingerprinting method is
very effective. Among the 10 devices, only one pair is near-collision. In
Figure~\ref{fig:fingerprinting}(b)-(d), we apply a policy to the 
accelerometer that rotates the device by a small angle $\theta$. In 
Figure~\ref{fig:fingerprinting}(b), $\theta$ is a random variable that is
uniformly distributed between $0$ and $10^{\circ}$. Similarly, $\theta$
in Figure~\ref{fig:fingerprinting}(c) and (d) is uniformly distributed 
between $0$ and $20^{\circ}$ and $30^{\circ}$, respectively. As shown, the more we rotate the 
device, the less distinct the estimation of $S_z$ and $O_z$ are for each 
device. The results in Figure~\ref{fig:fingerprinting}(c) are almost 
undistinguishable. Therefore, using \sysname's policy, we can 
prevent fingerprinting and keep the anonymity of mobile devices.

%\subsection{Usability}\label{sec-usability}
%
%\yanyan{user survey on privacy: show how people feel if their 
%privacy has been protected, whether device owners feel the 
%protection is enough.}
%
%\yanyan{incentives to participants}

\subsection{Performance Overhead}\label{sec-benchmark}

We measured the overhead incurred by running blurring layers as policies, 
along with experiment code in Sensibility Testbed. We show that it is 
feasible to protect the privacy on end-user devices without 
affecting user experience. 
%in terms of app responsiveness and battery life.

\begin{figure}
\center{\includegraphics[width=3in]{figs/time.pdf}}
\caption{\small Overhead incurred by running varying number of blurring 
layers with experiment code. Blue points show the execution time of 
an experiment using blurring layers without verifying the runtime behavior. 
Black points show the execution time with the runtime behavior verified. 
\yanyan{remove blue and add a base line?}
\label{fig-time}}
\end{figure}

%\textbf{Runtime overhead.}
Figure~\ref{fig-time} shows the runtime overhead of using varying number of 
blurring layers, i.e., policies, to protect sensor data. The experiment code 
calls \path{get_battery()} from the Repy sandbox. Each point in the 
figure is averaged over 100 iterations. Blue points show the execution time of 
using blurring layers without verifying the runtime behavior. Black points 
show the execution time with the runtime behavior verified, i.e., using a 
contract to check arguments, return values and exceptions. Note that a 
contract checks for both mutable and immutable data types~\cite{muta} 
for arguments and return values. When the number of layers is below 10, 
the runtime is slightly increased from 10$^{-5} s$ to 10$^{-4} s$. Although
the runtime overhead is higher with more layers, as Table~\ref{tab:policy} 
and \ref{tab:policy-continued} show, most projects need less than 10
policies. \yanyan{we don't mention contracts in paper anymore.}

Table~\ref{tab:overhead}(a) shows the total runtime overhead when an experiment
calls \path{get_accelerometer()}, without any blurring layer (no layer), with blurring 
layers but without verifying the runtime behavior (no-op), and with blurring layers
that verifies the runtime behavior and rounds up accelerometer to two decimal 
points (round-up). As shown in the table, the total runtime is \textit{linear} with an 
increasing number of layers/policies. The overhead per policy is almost
constant, with no-op round 0.12 - 0.14 $ms$ and round-up 0.14 - 0.21 $ms$. 
Therefore, the runtime overhead will not affect conducting an 
experiment or user experience.

Table~\ref{tab:overhead}(b) shows the memory overhead without any blurring layer 
(no layer), and with blurring layers that verifies the runtime behavior and rounds up 
decimal points (round-up). As the results show, the memory overhead is almost
constant, around 0.48 - 0.55 $MB$, until the number of blurring layers reaches 100. 
%\yanyan{is there an explanation?} 
Nevertheless, the overhead never exceeded 1 $MB$ in all cases. 

\begin{table}
\scriptsize
\centering

\bgroup
\def\arraystretch{1.15}% % for table padding

\begin{tabular}{|l|l|c|c|c|c|}
\hline
\multicolumn{2}{|c|}{\multirow{2}{*}{\bf Operations}} & 
\multicolumn{4}{c|}{\bf Number of blurring layers} \\\cline{3-6}
\multicolumn{2}{|c|}{} & {\bf 3} & {\bf 10} & {\bf 50} & {\bf 100}\\\hline

\multicolumn{2}{|c|}{No layer (baseline)} & \multicolumn{4}{c|}{6.276 $ms$} \\\hline

\multirow{2}{*}{No-op} & Total & 6.654 $ms$ & 7.756 $ms$ & 12.75 $ms$ & 18.94 $ms$ \\ 
& Per-layer & 0.1259 $ms$ & 0.1479 $ms$ & 0.1295 $ms$ & 0.1266 $ms$ \\\hline

\multirow{2}{*}{Round-up} & Total & 6.907 $ms$ & 7.991 $ms$  & 13.86 $ms$ & 20.64 $ms$ \\ 
& Per-layer & 0.2104 $ms$ & 0.1715 $ms$ & 0.1518 $ms$ & 0.1436 $ms$ \\\hline 
\multicolumn{6}{c}{\vspace{-0.1cm}}\\

\multicolumn{6}{c}{\small (a) Runtime overhead.} \\
\end{tabular}\\\vspace{0.25cm}
%\end{subtable}


\begin{tabular}{|l|l|c|c|c|c|}
\hline
\multicolumn{2}{|c|}{\multirow{2}{*}{\bf Operations}} & 
\multicolumn{4}{c|}{\bf Number of blurring layers} \\\cline{3-6}
\multicolumn{2}{|c|}{} & {\bf 3} & {\bf 10} & {\bf 50} & {\bf 100}\\\hline

\multicolumn{2}{|c|}{No layer (baseline)} & \multicolumn{4}{c|}{6.246 $MB$} \\\hline

\multicolumn{2}{|c|}{Round-up} & 6.742 $MB$ & 6.801 $MB$ & 6.727 $MB$ & 7.230 $MB$ \\\hline

\multicolumn{2}{|c|}{Overhead} & 0.492 $MB$ & 0.555 $MB$ & 0.480 $MB$ & 0.984 $MB$ \\\hline
\multicolumn{6}{c}{\vspace{-0.1cm}}\\

\multicolumn{6}{c}{\small (b) Memory overhead.} \\
\end{tabular}
%\end{subtable}

\egroup

\caption{\small Overhead with varying number of blurring layers. \yanyan{remove no-op in (a)?}}
\label{tab:overhead}
%\vspace{-10pt}
\end{table}


%\textbf{CPU and memory overhead.}
%\todo{what's the CPU/memory overhead?}

\subsection{Experience with \sysname}\label{sec-deployment}

%\todo{Q3: how is the testbed being used? A: deployment experience: 
%number of people, projects, issues, hackathon.}

\subsubsection{Experience Running Experiments}\label{sec-external}

Our early prototype of \sysname has been used by dozens of researchers.
This has come in two forms: researchers performing extended practical
deployments and researchers experimenting with Sensibility Testbed at a 
hack-a-thon.  As our IRB was recently approved, the experiments to date have 
focused on the usability of Sensibility Testbed as a platform for 
experimenters.
As a result, this section primarily answers the question: How easy is it to
use Sensibility Testbed?

{\bf Research deployment.}  Sensibility Testbed has been used by 
several research projects where the researchers installed Sensibility
Testbed on their own devices and then conducted an experiment.  
A research group in Austria created the Open3G \cite{open3g} project to
investigate cellular technology coverage in Vienna.  The researchers
who created this project estimated that the effort to write the code to
collect the sensor data took about an hour.

Sensibility Testbed was also used by a high school student as part of 
a vehicle data collection project~\cite{reininger2015first}.  The student
connected his phone to the on-board diagnostics sensor in a car and used Sensibility
Testbed to capture data.  The student then drove around the greater
New York area and used this information to derive information about driving
habits and patterns. 

%The experience from the 
%project developers was positive, and they identified cases where our 
%instructions for use were unclear. Despite these clarification issues, in 
%the case of Open3G, 
%the developer reported that writing the code to
%get cellular information took about an hour and they have used our
%testbed since 2012. We are currently in discussions with other 
%researchers about integrating Sensibility Testbed into their research
%projects, such as monitoring construction safety.

{\bf Hack-a-thons.}  In the past two years, we have hosted 
hack-a-thons co-located with 
the IEEE Sensors Applications Symposium (SAS)~\cite{sas}.  SAS attracts a 
diverse community of researchers that use sensors in their research.
The vast majority of participants are not computer scientists, but instead
come from other science disciplines.

We recruited SAS attendees to attend a hack-a-thon in each of the past two
years.  As a result, each year we had about twenty participants
spend a day of the conference building applications for the Sensibility Testbed.
None of the participants had any prior experience with 
Sensibility Testbed, and many of them also had no experience with Python.

As our IRB was not approved in time for these workshops, 
researchers implemented code that they then tested on their own devices.
%Each of the projects did a demonstration in the late afternoon session that 
%showed their application working in practice.  The team that built
%the best application received an award, including new Android phones.
%
Despite only having about six hours from they first learned about \sysname, 
researchers built some interesting and complex applications. These include 
navigating between conference rooms using WiFi connection information, 
monitoring battery information and turning off WiFi and Bluetooth when 
battery level of is low.

%\label{sec-ease}

%We are hosting another hack-a-thon with SAS 2016 and invite interested 
%researchers to join. \cappos{Likely cut}

\subsubsection{Experience Implementing Policies}\label{sec-implementing-policies}

In addition to experiments, 
%it is important that security policies can be implemented for Sensibility Testbed.  
the policy implementation mechanism in \sysname has 
been used in student assignments across ten classes at three institutions.
These assignments had students implement a policy to restrict access to 
some resource, usually the network or file system.  
%
In most classes, the students were given a week to learn the programming
environment and write their policy code.  Nearly all students successfully
completed the assignment, often stating in evaluations that implementing a new 
policy took a few hours.   More details about the implementation of policy
code in the classroom and our pedagogical experiences with them is available
in other sources~\cite{Cappos_SIGCSE_2014, Hooshangi_SIGCSE_2015}.
\yanyan{are these related to sensors?}
%Students were then
%provided every student's policy code and encouraged to try to violate the
%guarantees given in the policies.   The assignments were designed to force
%students to track filesystem or network state that is stored by the underlying
%kernel and to keep it synchronized 



%\cappos{Add content about Richard, Sara, and my experience with security
%policies in the classroom.}


