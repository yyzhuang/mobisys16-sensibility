\section{Introduction}

Today's end-user mobile devices, such as smartphones and
tablets, have become indispensable gadgets in people's everyday
life, and thus have created increasing research opportunities.
As a result, the Internet architecture, end-user traffic
patterns, etc., have also evolved rapidly. These devices
generate useful information for service providers and policy
makers to enhance network research and the services provided.
These devices also provide a platform for researchers to
investigate how new applications can provide better performance.
As a result, there has been significant interests in the network
community to study mobile network and devices, and in the
systems community to deploy new services and test research
prototypes~\cite{nikravesh2015mobilyzer}\yanyan{cite Mobilyzer, Phonelab, etc}.
					
For researchers, however, there are two obstacles for research
about mobile devices. First, the privacy and security challenges
have increased dramatically over the years. \yanyan{Add some
cites and examples.} For example, a smartphone's GPS locations,
WiFi connections, or Bluetooth pairing history can be highly
personal; a malicious party could also potentially bypass a
device's security protections and gain access to user
privileges. We not only need to ensure the security of a device
so that researcher's code cannot inadvertently damage or
maliciously hack into the device, but also protect the privacy
of device owners so that the code cannot eavesdrop on phone
conversations or infer passwords.

Second, it is challenging for researchers to perform meaningful
research related to end users without compromising data privacy
and ethical merit. Currently, research institutions adopt an
institutional review board (IRB) approval process where a
formally designated committee reviews, approves, and monitors
research involving human subjects [cite
https://en.wikipedia.org/wiki/Institutional\_review\_board ].
Due to the different goals and unforeseen risks of each
individual project, the researcher needs to recruit subjects for
each research study, obtain their informed consent, etc. This
process is time-consuming, and the researcher needs to repeat
the process for every different projects. Even with the
recruited participants, the researchers from different research
groups cannot test their hypothesis at a world-wide scale or
reuse each other's user base.
					
This work is a first step towards lowering the technical
barriers to mobile Internet research without lowering the
ethical standard [4], while protecting the security and privacy
of end users. We design and implement Sensibility Testbed [2,
5], an Internet-wide measurement testbed for mobile devices that
allows researchers to run code and deploy services on ordinary
people's smartphones or tablets for research purpose. It ensures
the security of user-owned devices and the privacy of
user-generated data. The usage model of Sensibility Testbed is
unique in that it manages how device owners make their devices
accessible to different research communities without putting
their devices at risk. Meanwhile, it offers technical measures
that allow researchers to collect data from remote mobile
devices without impairing the device owner's privacy. The
Sensibility Testbed relieves researcher from the burden of
recruiting subjects for every single experiment; the device
owners need only give consents once, instead of giving multiple
consents to each project of each researcher.

In Sensibility Testbed, there are three types of interacting
parties: mobile devices owned by ordinary people, with our app
installed; a clearinghouse server that discovers and configures
participating devices; and researchers wanting to run
experiments on mobile devices (see Figure 1). Mobile devices
provide resources and data for researchers to use in their
experiments. In order to perform safe experiment on mobile
devices, a researcher must provide to the clearinghouse server
the IRB policies from his institute for accessing devices. The
clearinghouse server helps the researcher acquire and manage
devices, and also codifies the policies specified by the
researcher's IRB into data blurring layers that are enforced on
mobile devices (Section XX). Such a process can protect device
owners' personal information. After obtaining remote sandboxes
and having IRB policies in place, researchers can perform
experiment on the devices using their credentials assigned by
the clearinghouse. Researchers' code runs in a sandbox on any
remote device that isolates the code from the rest of the device
host system. To control the execution of code, the researcher
uses a local machine to manage the experiments via an experiment
tool (ET). This tool can deploy and run experiments in sandboxes
on remote devices that are acquired through the clearinghouse.


Figure 1: Sensibility Testbed architecture. 

Using these techniques, Sensibility Testbed makes experiment
prototyping faster, the remote control and management of devices
easier, and running experiment code more secure. The
contributions of this work are as follows:
