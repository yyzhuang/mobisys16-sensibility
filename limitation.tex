\section{Practical Challenges and Limitations}\label{sec-limitation}

Below we discuss potential deficiencies of Sensibility Testbed 
that should be kept in mind when designing experiments to run on it, 
and consider various challenges we met while building the system.

\subsection{Limitations}\label{subsec-limitations}
In Sensibility Testbed, experimenter code runs inside a sandbox, and 
can access sensors only through privacy-preserving blur layers. 
In contrast to a native Android research app, this starkly curtails 
the experiment's access to the Android device. For example, Sensibility 
Testbed does not provide a way to interact with the device owner 
through unsolicited on-screen messages or notifications that a native 
app could send\footnote{Interactivity still is possible if the device 
owner chooses to make the first move: An experiment may serve a web 
page on the local device through the web browser, and interact with 
the device owner as a ``web app''.}. 

On similar terms, Sensibility Testbed limits access to other parts 
of the Android API. For example, there is no way for an experiment to 
signal that it wants to handle \texttt{Intent}s of any sort, e.g. 
act as an MP3 player whenever the device owner wants to open a file 
of this type, or handle phone calls, etc.
\albert{Playing back audio files, handling calls and text messages etc. 
are examples of things implemented via Intents in Android, see 
\url{https://en.wikipedia.org/wiki/Intent\_\%28Android\%29} and 
\url{https://developer.android.com/reference/android/content/Intent.html}. 
Do you think we should explain the concept of Intents more?}

It should be noted that the operations in the above examples 
could well be added to the sandbox API -- there is no technical 
obstacle for this. Rather, it was our fundamental security design 
choice for Sensibility Testbed to not allow these.


\subsection{Challenges}\label{subsec-challenges}

In addition to the limitations discussed above, further questions 
arise from currently existing challenges. We hope to be able to 
answer these questions as we gain more practical experience with 
Sensibility Testbed.  
We approach questions of use, usefulness, technical design, and
general architecture, in this order.


\textbf{Use.}~
From the perspective of a researcher designing an experiment to 
run on Sensibility Testbed, there is a certain learning phase 
to go through in order to familiarize with the test and deployment 
tools, the sandbox API, operational aspects of a distributed system, 
and so on. Since the sandbox uses a different programming language 
than Android usually does, and the patterns to access sensor values 
are different as well, no previous experience with Android is required 
to program for Sensibility Testbed. Indeed, from our experience in 
educational contexts, an experimenter who brings along any programming 
knowledge will quickly learn how to use the sandbox, while specific 
Android experience does not necessarily provide an advantage.
% Also, the sandbox API is small, and stays away from unconventional 
% or OS-specific patterns, so it is easy to learn even for the
% uninitiated.


\textbf{Usefulness.}~
Sensibility Testbed's current feature set probably makes it most 
useful as a passive, distributed, large-scale sensor network. 
It does not map well to interactive crowd sensing, crowd sourcing, 
and (geo) tagging tasks, for studies looking at personal data stored 
on the phone (such as performing research using the device owner's 
phone book),
% Cf. https://en.wikipedia.org/wiki/Six_degrees_of_separation
or introspection into the Android OS (which, e.g., PhoneLab provides). 
In short, we expect other approaches to remain valid and interesting 
besides Sensibility Testbed, but hope that its scale and distribution 
are attractive properties nevertheless.

\albert{Could add questions regarding scale, too:
``Is S.T. a meaningful way to scale out smartphone research?'',
``Shall we scale it out at all, or let Google / the mobile operators 
do it''.}

Another interesting aspect we hope to be able to understand better as 
we open Sensibility Testbed to the public is what motivates device 
owners to altruistically ``donate'' resources to researchers. 
We hypothesize that altruism is an incenctive, especially since 
Sensibility Testbed currently provides no native (GUI-based, 
notification) way for experiments to interface with the device 
owner. We note that there are projects that keep records of 
resource donations (like a high-score table), let contributors form 
teams, and so on\footnote{See OpenWLANMap for a concrete example, \url{http://www.openwlanmap.org/owlmapatandroid.php?lang=}}.
% With due thanks to Karl Karpfen for pointing out the project!
This rewards contributors with some form of acknowledgment and 
publicity.
Other projects appear to have utilized ipure altruism to great effect.
For example, \textit{that ol' SIGCOMM ``middleboxes''} paper 
\cite{it} claims \textit{tons of} installs from the Android app 
store, suggesting that there exists a sizable population of device 
owners that will install an app that gives them neither an obvious, 
direct benefit from interactivity, nor a form of publicity like 
the high-score table.


\textbf{Technical challenges.}~
Sensibility Testbed is designed to minimize the privacy repercussions 
of smartphone research by blurring the sensor values that experiment 
code sees. For this, we provide blur layers that transform the sensor 
values, round them or otherwise limit their accuracy, replace them by 
random data, limit the access rate, and so on (see \S\ref{...}). 
In \S\ref{...} we show that blurring is a meaningful strategy to 
counterbalance a researcher's desire for sensor access and a device 
owner's desire for privacy, and that our implementation of blurring 
can express complex interrelations between sensor values and accuracies. 
Still, this does not guarantee that the existing blurring layers and 
configurations will always be able to address adequately all possible 
privacy concerns. Also, bugs in the sandbox or blur code might 
expose the device owner's privacy.  
To meet these objections, Sensibility Testbed includes a software 
updater copmponent so that we can fix any security issues and 
improve the coverage provided by the blur layers.


\begin{itemize}
  \item Are blur layers the right way to solve the problem? E.g., 
collect enough data to make use of temporal autocorrelation; secure enough.
  \item Granularity of user interaction is on opt-in / opt-out 
level. No way currently for users to add own blur, cf. Sensorium 
which does have this.
  \item Are IRBs the right way to solve the problem? E.g., 
rogue researchers that deliberately obfuscate what they will 
do with sensor readings, so as to breach privacy. E.g., 
ignorant IRBs that greenlight everything and put the burden on 
our default IRB (????).
\end{itemize}
