\section{Testbed Operation}

\subsection{Clearinghouse tracks devices}\label{sec-case1}
Once the app is started, Alice's device can be discovered by 
the clearinghouse. The resource manager in her app periodically contacts 
a lookup service to advertise the device sandbox to the rest of the testbed, 
such as the clearinghouse or experiment manager. \yanyan{resource 
manager and lookup service have not been introduced yet.}
A lookup service is a distributed key-value store, such as a 
distributed hash table (DHT)~\cite{dht}, that allows one to associate 
keys with values and retrieve values associated with keys. In this case, 
Alice's resource manager uses an \textit{identification key}, \path{key.alice}, 
generated during the app installation, and associates this key 
with Alice's device IP address and port number, by 
storing this key-value entry in the lookup service. This key is not 
associated with Alice or her device's identity, but only with the app's 
installation on the device. 
%When the clearinghouse knows Alice's key, it can retrieve Alice's 
%IP:port by looking up her key, \path{key.alice}, through 
%the lookup service. With IP:port, the clearinghouse can then
%communicate with Alice's device over the network.
\yanyan{I think the device first puts key.sensibility 
in the lookup service. is this key's value key.alice? so when the 
clearinghouse looks up key.sensibility, it finds out key.alice; and then
it uses key.alice to lookup IP:port. But is it necessary to go into such detail?}

%The app interface also displays a list of experiment running
%on Alice's device, and their policies. Therefore, 
%through the app interface, Alice may uninstall, or 
%choose to opt out of individual experiments. 
%\lois{how? I have not seen this explained yet and I think its critical to discuss this}

To keep track of Alice's device, the
clearinghouse periodically queries the lookup service to
discover any new device installs. Once Alice's device is discovered, the
clearinghouse obtains \path{key.alice}, and thus
%clearinghouse uses a database that stores her device's unique
%identification key, \path{key.alice}, generated during installation. 
can retrieve Alice's  IP address and port number by 
looking up her key in the lookup service. This allows the clearinghouse
to communicate with Alice's device over the network.
If Alice uninstalls the Sensibility Testbed app, 
\path{key.alice} is deleted at the clearinghouse, which effectively unlinks
her device from any metadata stored on the clearinghouse. The key
will also be removed from the lookup service after a period of inactivity.
An experiment manager controlled by a researcher can lookup
available device installs in a similar way. Instead of finding all devices
available to the testbed, the experiment manager will only be able to 
lookup devices assigned to the researcher by the clearinghouse.

%The clearinghouse
%plays an intermediate role between the experimenter and 
%the device owner.
%As described in Section~\ref{sec-overview}, when an 
%experimenter registers at the clearinghouse, he
%needs to provide his IRB policies. These policies ensure that
%the researcher cannot conduct experiments to access data that
%extend beyond the experiment policy. The clearinghouse 
%translates and codifies each policy, and instructs the 
%sandboxes on remote devices to implement these policies. 
%When experiment code is running in the sandbox, the 
%policies will be applied to restrict %the precision of sensor 
%%data or the frequency to access 
%sensor access. 